Neste trabalho foi possível entender os fundamentos de aprendizado supervisionado, as etapas da construção de um modelo de aprendizado não supervisionado, o problema de classificação, as métricas de validação de performance, os hiperparâmetros e a interpretabilidade do modelo.

Conseguimos entender a diferença entre os algoritmos XGBoost, CatBoost e LightGBM e aplicar suas implementações em conjuntos de dados reais da medicina. Foi feita uma analise descritiva dos conjuntos de dados e aplicado as transformações para correção dos dados e/ou transformação das variáveis categóricas para numéricas para os modelos de XGBoost e LightGBM.

No final, foi possível entender o impacto dos hiperparâmetros no modelo através do Optuna. Foi possível encontrar quais hiperparâmetros tiveram a maior importância no estudo nos diferentes conjunto de dados. Foi criada uma função sólida em \textit{.py} para o treino, otimização e estudo do Optuna, sendo possível utilizar nos três algoritmos sem nenhum problema. Analisando a principal métrica de validação, \textbf{AUC}, tivemos um valor de ganho de performance para os modelos tunados pelo Optuna em relação aos modelos com os hiperparâmetros \textit{Default} entre 5\% e 10\% para os conjunto de dados de Diabetes, Insuficiência Cardíaca e Carcinoma de Mama. Vale ressaltar que, para o modelo mais complexo de prever, Insuficiência Renal, tivemos ganhos de 20-30\%. E no final, conseguimos interpretar como cada variável funcionou no modelo pelo SHAP, o que é extremamente importante para conseguir explicar o modelo.

\section{Limitações e Trabalhos Futuros}
Uma das principais limitações deste trabalho foi o tempo de execução. Treinar os modelos pelo Optuna e para um grande espaço de hiperparâmetros levou tempo. Foi possível utilizar uma escala logarítmica para alguns hiperparâmetros para gerar resultados com maiores diferenças de magnitude, então para futuros trabalhos a sugestão é aumentar a escala dos hiperparâmetros e estudar mais casos que o Optuna nos retornou com um número maior de conjunto de dados.